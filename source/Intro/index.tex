\section{Fear}

Every time I introduce myself at a party, my new acquaintance asks,
``So, what do you do?''  ``I'm a physicist. I teach physics and math
at a university,'' I say, and then I pause, because just about every
time, the reaction is the same: ``Oooo --- you must be very smart.  I
was \emph{terrible} at that.''  And since I'm a slightly awkward geek,
I stammer out my assurances that I'm not that smart, and then I
commiserate about how difficult the high school/college versions of
science or math classes usually are. 


I do a bit better with my students, where I have more time to talk,
but the results are frequently the same: A large percentage of the
people I meet claim to have liked math and science at some point in
their lives, they even did well at it once upon a time.  But then
something happened.  When that something occurred shows a lot of
variation: in middle school, in high school, in college.  But at some
point, most people decide that science and math are not for them --
they feel that they have been \emph{told}, rudely, that they are not
\emph{good enough} for math and science.  They want to work hard in my
class, they assure me, but I have to understand that this stuff just
doesn't come naturally to them.


So I start every term with a problem, as soon as the students walk in
the room.  It goes something like this:

\subsection{Goats and Sports Cars}
Note: This is a famous problem in mathematics.  If you've seen it
before and know the answer, keep it to yourself for now.  If you
haven't seen it before or you don't know the answer, DON'T LOOK IT UP!
The point isn't the answer, it is the process of getting the answer.

Suppose you are on a game show, and the host offers you an opportunity
to select a prize from behind one of three doors.  The prize (a sports
car) is behind one of them.  There are goats behind the other two.  So
you select one of the doors.

Now stop right there for a moment, and think about the choice you just
made.  What is the probability that you were right on your first
guess?  Don't click on the ``Answer'' link below until you are
satisfied that you have an answer.  You can use the computation box
below to do some figuring, if you have to.


\begin{sageverbatim}
#You can use this cell like a calculator.  
#Type in a calculation, and hit ``Shift-enter'' to evaluate.
\end{sageverbatim}


%right here I want a ``hidden'' sage cell, where the hidden tag is
%replaced with ``answer''.
%TODO: Change the display of the word ``hide'' to ``answer''.
\begin{sageverbatim}
%hide
%latex
#The answer is $\frac{1}{3}$, since there are three choices, and one has the prize.
\end{sageverbatim}


But now, like any good game show host, I'm going to throw you a
curveball: I don't want the game to end so quickly.  There's not
enough time run more commercials, or to play that awful music, unless
I make you make you sweat some more. 


So here's the play: I announce that I know where the prize is, and I'm
going to help you out by showing you that the prize is NOT behind one
of the doors you didn't choose (I wouldn't expose the one you chose,
because then the game would be over.  No Fun.  And I won't expose one
where the prize IS, because, again, No Fun. And since we know that
there's only one prize, I can always find a door to open that won't
ruin the game.  Let's say you guessed Door Number 1, and then I open
Door Number 3, and show you that there's a goat back there.  And then
I ask, with as much cheesiness as I can muster: ``Would you like to
switch?  To Door Number 2?  Or would you like to stick with your
original guess of Door Number 1?''




%right here I want a sage cell that can be computed in
\begin{sageverbatim}
%hide
%auto
print "n=1:\t     1 == 1(1+1)/2\t", 1 == 1*(1+1)/2  
print "n=2:\t 1 + 2 == 2(2+1)/2\t", 1 + 2 == 2*(2+1)/2
\end{sageverbatim}

\begin{sageverbatim}
%hide
%auto
@interact
def _(n=(1..100)):
  summation = " + ".join([str(x) for x in range(1,n+1)])
  product = n*(n+1)/2
  html('Left: $%s\;=\;%s$'%(summation,sum(range(1,n+1))))
  html('Right: $\\frac{n(n+1)}{2}\;=\;\\frac{%s(%s+1)}{2}\;=\;%s$'%(n,n,product))
\end{sageverbatim}











