\section{Fear}

Every time I introduce myself at a party, my new acquaintance asks,
``So, what do you do?''  ``I'm a physicist. I teach physics and math
at a university,'' I say, and then I pause, because just about every
time, the reaction is the same: ``Oooo --- you must be very smart.  I
was \emph{terrible} at that.''  And since I'm a slightly awkward geek,
I stammer out my assurances that I'm not that smart, and then I
commiserate about how difficult the high school/college versions of
science or math classes usually are. 


I do a bit better with my students, where I have more time to talk,
but the results are frequently the same: A large percentage of the
people I meet claim to have liked math and science at some point in
their lives, they even did well at it once upon a time.  But then
something happened.  When that something occurred shows a lot of
variation: in middle school, in high school, in college.  But at some
point, most people decide that science and math are not for them --
they feel that they have been \emph{told}, rudely, that they are not
\emph{good enough} for math and science.  They want to work hard in my
class, they assure me, but I have to understand that this stuff just
doesn't come naturally to them.


So I start every term, as soon as the students walk in the room, with a
problem, it goes something like this:

\subsection{Goats and Sports Cars}







