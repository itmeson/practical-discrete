\chapter{Introduction}

\section{Fear}

Every time I introduce myself at a party, my new acquaintance asks,
``So, what do you do?''  ``I'm a physicist. I teach physics and math
at a university,'' I say, and then I pause, because just about every
time, the reaction is the same: ``Oooo --- you must be very smart.  I
was \emph{terrible} at that.''  And since I'm a slightly awkward geek,
I stammer out my assurances that I'm not that smart, and I
commiserate about how difficult the high school/college versions of
science or math classes usually are. 


I do a bit better with my students, where I have more time to talk,
but the results are frequently the same: A large percentage of the
people I meet claim to have liked math and science at some point in
their lives, they even did well at it once upon a time.  But then
something happened.  When that something occurred shows a lot of
variation: in middle school, in high school, in college.  But at some
point, most people decide that science and math are not for them --
they feel that they have been \emph{told}, rudely, that they are not
\emph{good enough} for math and science.  They want to work hard in my
class, they assure me, but I have to understand that this stuff just
doesn't come naturally to them.


So I start every term with a problem, as soon as the students walk in
the room.  It goes something like this:

\subsection{Goats and Sports Cars}
Note: This is a famous problem in mathematics.  If you've seen it
before and know the answer, keep it to yourself for now.  If you
haven't seen it before or you don't know the answer, DON'T LOOK IT UP!
The point isn't the answer, it is the process of getting the answer,
and if you look it up, you will short circuit the process of thinking
about it.

Suppose you are on a game show, and the host offers you an opportunity
to select a prize from behind one of three doors.  The prize (a sports
car) is behind one of them.  There are goats behind the other two.  So
you select one of the doors.

Now stop right there for a moment, and think about the choice you just
made.  What is the probability that you were right on your first
guess?  Don't click on the ``Answer'' link below until you are
satisfied that you have an answer.  You can use the computation box
below to do some figuring, if you have to.


\begin{sageverbatim}
#You can use this cell like a calculator.  
#Type in a calculation, and hit ``Shift-enter'' to evaluate.
\end{sageverbatim}


%TODO: I want to be able to change the display of the word ``hide'' to ``answer''.
\begin{sageverbatim}
%hide
%auto
@interact
def _(Probability=([0/3,1/3, 2/3, 3/3]), auto_update=False):
    choice = Probability
    if choice == 1/3:
         html('1 / 3 is right.  There are three doors, and the prize is behind one of them.')
    else:
         html('%s is not right.'%(choice))
\end{sageverbatim}    


But now, like any good game show host, I'm going to throw you a
curveball: I don't want the game to end so quickly.  There's not
enough time run more commercials, or to play that awful music, unless
I make you make you sweat some more. 


So here's the play: I announce that I know where the prize is, and I'm
going to help you out by showing you that the prize is NOT behind one
of the doors you didn't choose (I wouldn't expose the one you chose,
because then the game would be over.  No Fun.  And I won't expose one
where the prize IS, because, again, No Fun. And since we know that
there's only one prize, there's always a door available to open that won't
ruin the game.  Let's say you guessed Door Number 1, and then I open
Door Number 3 to show you that there's a goat back there.  And then
I ask, with as much cheesiness as I can muster: ``Would you like to
switch?  To Door Number 2?  Or would you like to stick with your
original guess of Door Number 1?''


What do you do?  Is it better to switch, or to stay?  Or does it even
matter?


I recommend trying it out.  Write down your best guess as to the
correct strategy (the strategy that leads you to win the sports car
most often) and an explanation of WHY that strategy is the best one.
Then play the game.  Pretend to be the host, have a friend be the
contestant, and keep track of how many times you win if you stay with
your original guess, and how many times you win if you switch.


\subsection{The answer}
Alright, so I assume you tried it out, and since this is a game that
involves some randomness, I hope you tried it a bunch of times,
because we all know that it is possible to flip ``Heads'' 10 times in
a row (thought still pretty unlikely) -- but it is only when you try it
hundreds of times that you can be sure the pattern of half and half
will be observed.


What did you find?


Almost everyone makes an initial answer that it doesn't matter which
door you pick.  The reasoning is incredibly compelling: There are two
doors.  We have no reason to favor one over the other, therefore the
prize has an equal probability to be behind either door, therefore
there is no advantage in switching or in staying.


Incredibly compelling, but wrong.


Once you do the experiment by playing the game enough times, almost
everyone becomes convinced that in fact there is a $\frac{2}{3}$rds
chance of winning if you switch, and only $\frac{1}{3}$rd if you stay,
because you win more than 60\% of the time if you switch, and only
around 30\% of the time if you stay.


And here's the interesting thing: most people still have a hard time
putting into words why the answer SHOULD BE that it is better to
switch.  They can accept, empirically, that it seems to come out that
way, but lacking a good explanation of why, many people come away from
this problem confused, or even refusing to believe the result.


I personally took a couple of weeks to understand this problem, when I
first saw it in Discrete Mathematics as an undergraduate.  The reason
the problem is famous is that it was discussed in a popular magazine
article in the early 1990's, and dozens of mathematics professors
wrote to the author, (rudely) claiming that she had gotten it wrong
when in fact she was right.  Most of them eventually wrote sincere
apologies.


I'm not going to go into an explanation of the correct answer here,
because I've found that it mostly doesn't help people to \emph{hear}
(or read) the answer.  It helps them to DO it.  And then think through
it, and make sense of the problem themselves.


Instead, what I want to do is talk about what this problem, and
people's reactions to it, tells us about Mathematics.

\begin{itemize}
\item \textbf{When you do math, you will be WRONG most of the time.}

  Most people have a really hard time with being told they are wrong.
  And somehow math teachers are, let's say, \emph{gifted} at making
  other people feel stupid.  It is not intentional, I assure you.  I
  am a math teacher, and a lot of my friends are math teachers.  And
  as a group, we are usually well-meaning, friendly, thoughtful people
  who really want to help you succeed.  But somehow we come across
  condescending jerks.


  I think this happens because of Fact 1.  It is very easy to tell
  when something is wrong in math.  The numbers and equations provide
  a handy truth-check that makes nonsense just leap off the page, once
  you know what to look for.  Math teachers see that kind of thing
  every single day, so we are very good at spotting errors.  And
  pointing them out.  And that makes people feel dumb.  And no one
  likes to feel dumb.  Therefore, lots of people don't like math.


  But being wrong most of the time is actually one of the best
  features of math.  One doesn't improve without knowing that
  something needs improving.  Math helpfully points out when we are
  wrong, so we can improve.  This may sound like a motivational
  platitude, but I assure you, that even as a working scientist, I am
  wrong MOST of the time, and days when I find that I wasn't wrong are
  usually days in which I wasn't working very hard.


\item \textbf{When you try to figure something out, it is often best
    to try an experiment first.}  

  Understanding the Monty Hall problem just by thinking about it
  doesn't work for most people, and there is NOTHING WRONG WITH THAT.
  Many people think that if they don't ``get'' something quickly, it
  must be too hard for them, or they take it as evidence that this is
  not their calling.

  Math is hard for everybody (some of us have just forgotten about all
  the work we once did), and it's silly to think that you can ``think
  your way through'' everything.  Doing experiments, drawing pictures,
  talking to others, finding ways to visualize and think about a
  problem, and getting it wrong -- all of these are essential parts of
  actually figuring stuff out.

  This book will help you learn more tools to understand and figure
  out problems, sometimes using techniques that may seem like tedious
  distractions.  You will find yourself wondering, ``Why won't they
  just teach me the \emph{right} way to do it and be done with it,
  instead of forcing us to do all this exploration?''

  The sad truth is that as professors, we are no more able to
  anticipate the kinds of problems you will actually face than we are
  to predict the future.  I couldn't possibly teach you every single
  algorithm, so you could follow it like a cookbook, even if I wanted
  to.  If I did, where would you be when you were faced with your
  first genuine problem, with no idea how to begin?

  And, to be more blunt, if I could put it in a cookbook, then why
  wouldn't I just let a computer do it, rather than try to teach
  fallible humans who will always be slower?

\item \textbf{Don't memorize too much.  But do memorize some}

  You are much better off to learn the big guiding principles at play,
  and then to practice applying them so you can understand their power
  and limitations.  Don't memorize the steps to a particular proof --
  instead memorize the ideas that inspired it.  As a student, this is
  of course very hard to do, so it is my job to help point them out,
  and to give you exercises that will stretch you in just the right
  ways.

\item \textbf{Develop a process that can help you attack new problems}

  I'm going to focus a lot on a particular heuristic approach that
  will be helpful in dealing with a lot of problems.  It won't always
  work, and you won't always need all of it, but you should practice
  thinking about your process, so you always have a toolkit handy even
  when faced with something that is totally new.  That toolkit, along
  with your intuition, is your best weapon in keeping up with the
  machines.

  My own process can be summarized with the following 5 steps:
  Experiment, Visualize, Conjecture, Prove, Refine.  The most
  important part of each step is that every one of them can be used to
  show that I am wrong.  I have an idea about a problem?  Conduct an
  experiment.  It might turn out that I was wrong.  I think of a way
  to visualize the process at play in a problem, using a drawing, a
  graph, a bunch of dice, some carboard boxes -- something.  I might
  find out that I was wrong.  I think I have a strong conjecture about
  the behavior of a mathematical object, so I try to construct an
  argument to prove it -- it might turn out that I was wrong, and the
  process of constructing that logic will help me see where.

  If you are doing something that cannot possibly lead to you
  discovering that you were wrong -- then you are not doing
  mathematics.

\item \textbf{Wrong is good}

 In case I haven't been clear enough: being wrong is good.  It means
 that you are thinking, it means that you are being challenged.  Most
 people feel stupid because math points out how often they are wrong.
 Instead, they should feel empowered because it proves that they are
 doing something with consequence.

\end{itemize}










