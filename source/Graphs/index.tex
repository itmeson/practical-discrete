\chapter{Graphs}

\section{Introduction}

Let's start with a silly question:  how many friends are there?
What's silly about it?  Well, what does it even mean?  Let's make it
more concrete:

\begin{sageverbatim}
%hide
%auto
M = Matrix ([ [0, 1, 1, 1, 1, 1],   #a 6x6 adjacency matrix
              [1, 0, 1, 0, 1, 0], 
              [1, 1, 0, 0, 1, 1],
              [1, 0, 0, 0, 0, 0],
              [1, 1, 1, 0, 0, 1],
              [1, 0, 1, 0, 1, 0] ] )
G = Graph(M)
G.show()
\end{sageverbatim}

the object above is called a \textbf{graph}.  The numbered points
represent people.  The lines connecting those points represent
friendships.  ``Person 0'' is popular.  She has five friends --
everybody else.  Person 3 is not quite so popular, he only has one
friend.  How many friends do each of the other people have?

\begin{sageverbatim}
%hide
%auto
@interact
def _(t1=text_control("Enter the number of friends for each person:"), auto_update=False, 
    d0 = input_box(default=0, label="Person 0", type=Integer),
    d1 = input_box(default=0, label="Person 1", type=Integer),
    d2 = input_box(default=0, label="Person 2", type=Integer),
    d3 = input_box(default=0, label="Person 3", type=Integer),
    d4 = input_box(default=0, label="Person 4", type=Integer),
    d5 = input_box(default=0, label="Person 5", type=Integer)):

    degrees = [d0, d1, d2, d3, d4, d5]
    degrees = [int(x) for x in degrees]
    d_act = [G.degree(x) for x in G.vertices()]
    if d_act != degrees:
        print "One or more of these is wrong: ", degrees
    else:
        print "All correct."
\end{sageverbatim}

%TODO: Have to have in-document references working.
Write a sentence explaining how you can compute the number of friends
a person has, if you are given a graph like above.

\begin{sageverbatim}
#Enter your answer here

\end{sageverbatim}

What do you get if you add up the total numbers of friends from each person? In
the graph above you get:

\begin{sageverbatim}
#enter your answer here

\end{sageverbatim}

Now do the same for the following graph (find the number of friends
for each person, and compute the total number of friends by adding up
the total for each person):
\begin{sageverbatim}
%auto
G2 = graphs.RandomGNP(n=10, p=0.5)
G2.show()
\end{sageverbatim}

This graph is a little bigger, and was generated randomly (a random
process was used to decide which people should be connected to which
other people).  Create a few graphs yourself by running the cell
below.  You can select different values for n (the number of people)
and p (the probability that a given pair of people will happen to be
friends).

\begin{sageverbatim}
%hide
%auto
@interact
def g3graph(n=slider(vmin=3, vmax=30, step_size=1, label='n'), p=slider(vmin=0.5,vmax=1.0,step_size=0.1, label='p')):
    G3 = graphs.RandomGNP(n, p)
    G3.show()
\end{sageverbatim}

You can get Sage to tell you the number of friends for a particular person by entering \textbf{G3.degree(5)}
to get the number of friends for person 5.  Try it.

%TODO: objects defined inside interacts are not globally available!!
\begin{sageverbatim}
print G3.degree(5)
\end{sageverbatim}




In the language of graphs, the number of friends a person has is also
called the \textbf{degree} of the corresponding \textbf{vertex}.












