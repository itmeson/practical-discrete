\chapter{Graphs}

% helper functions

\begin{sageverbatim}
%hide
%auto

def graph_highlight_edge_set(g, list_of_edges, color="red", default="black"):
    """Color a subset of edges of a graph a specified color."""
    
    x_list = [(e[0],e[1]) for e in list_of_edges]
    for e in g.edges():
        g.set_edge_label(e[0],e[1], 1 if (e[0],e[1]) in x_list or  (e[1], e[0]) in x_list else 2)
    d = {1: color, 2: default}
    print g.edges(), d
    g.plot(edge_colors=g._color_by_label(d)).show()

\end{sageverbatim}



\section{Introduction}

Let's start with a classic question -- consider the following image of a city.
A river runs through the city center, and there are two islands in the river.
A total of 7 bridges connect the islands to each other and to the banks of the
river.  Supposedly the inhabitants of this city speculated on whether it was
possible to take a tour of the city, visting every bridge exactly once without
repeating any.  No one had ever succeeded in doing it, but there are a lot of possible paths one could take, and it was not obvious whether it could be done or not.  

%TODO: How do you embed external images?

Before we get into the details, take a few minutes and try to do it yourself --
use pencil and paper and try to trace a path that hits every bridge without
repeats.


\begin{sageverbatim}
%hide
%auto
M = Matrix ([ [0, 2, 0, 1],   #a 4x4 adjacency matrix
              [2, 0, 2, 1], 
              [0, 2, 0, 1],
              [1, 1, 1, 0] ] )
global G_Bridges
G_Bridges = Graph(M)
G_Bridges.show()
\end{sageverbatim}



The answer isn't obvious, is it?  It \emph{seems} like there isn't a way to do
it -- one always seems to get stuck on one of the land areas, unable to get
back to one of the bridges.  But if we're going to be confident in asserting
that there isn't any way to do it, we're going to have be a lot more concrete
than that.  Historically, this problem was the genesis of the entire field of
graph theory.  In 1735, Leonhard Euler, one of the truly staggering
mathematical intellects of history, published a paper demonstrating that it is in fact impossible.  His argument was  


Let's start with a silly question:  how many friends are there?
What's silly about it?  Well, what does it even mean?  Let's make it
more concrete:

\begin{sageverbatim}
%hide
%auto
M = Matrix ([ [0, 1, 1, 1, 1, 1],   #a 6x6 adjacency matrix
              [1, 0, 1, 0, 1, 0], 
              [1, 1, 0, 0, 1, 1],
              [1, 0, 0, 0, 0, 0],
              [1, 1, 1, 0, 0, 1],
              [1, 0, 1, 0, 1, 0] ] )
G = Graph(M)
G.show()
\end{sageverbatim}

the object above is called a \textbf{graph}.  The numbered points
represent people.  The lines connecting those points represent
friendships.  ``Person 0'' is popular.  She has five friends --
everybody else.  Person 3 is not quite so popular, he only has one
friend.  How many friends do each of the other people have?

\begin{sageverbatim}
%hide
%auto
@interact
def _(t1=text_control("Enter the number of friends for each person:"), auto_update=False, 
    d0 = input_box(default=0, label="Person 0", type=Integer),
    d1 = input_box(default=0, label="Person 1", type=Integer),
    d2 = input_box(default=0, label="Person 2", type=Integer),
    d3 = input_box(default=0, label="Person 3", type=Integer),
    d4 = input_box(default=0, label="Person 4", type=Integer),
    d5 = input_box(default=0, label="Person 5", type=Integer)):

    degrees = [d0, d1, d2, d3, d4, d5]
    degrees = [int(x) for x in degrees]
    d_act = [G.degree(x) for x in G.vertices()]
    if d_act != degrees:
        print "One or more of these is wrong: ", degrees
    else:
        print "All correct."
\end{sageverbatim}

%TODO: Have to have in-document references working.
Write a sentence explaining how you can compute the number of friends
a person has, if you are given a graph like above.

\begin{sageverbatim}
#Enter your answer here

\end{sageverbatim}

What do you get if you add up the total numbers of friends from each person? In
the graph above you get:

\begin{sageverbatim}
#enter your answer here

\end{sageverbatim}

Now do the same for the following graph (find the number of friends
for each person, and compute the total number of friends by adding up
the total for each person):
\begin{sageverbatim}
%auto
G2 = graphs.RandomGNP(n=10, p=0.5)
G2.show()
\end{sageverbatim}

This graph is a little bigger, and was generated randomly (a random
process was used to decide which people should be connected to which
other people).  Create a few graphs yourself by running the cell
below.  You can select different values for n (the number of people)
and p (the probability that a given pair of people will happen to be
friends).

\begin{sageverbatim}
%hide
%auto
@interact
def g3graph(n=slider(vmin=3, vmax=30, step_size=1, label='n'), p=slider(vmin=0.5,vmax=1.0,step_size=0.1, label='p')):
    global G3 
    G3 = graphs.RandomGNP(n, p)
    G3.show()
\end{sageverbatim}

You can get Sage to tell you the number of friends for a particular person by entering \textbf{G3.degree(5)}
to get the number of friends for person 5.  Try it.

\begin{sageverbatim}
#run this cell, try changing the number in the parentheses, and observe how it relates to the graph above.

print G3.degree(5)

# Write a brief explanation of what this command does, and what 'degree' means.
\end{sageverbatim}

Using the degree command, figure out how many friends each person in
your graph has, then add them up to get the total number of friends.
Enter it below to check your work:

%TODO: Why do I have to hit Update twice?
\begin{sageverbatim}
%hide
%auto
@interact
def check_friends(t1=text_control("Enter the total number of friends:"), auto_update=False, 
    counted_friends = input_box(default=0, label="Total", type=Integer)):
    G3_friends = sum(G3.degree())
    if G3_friends == counted_friends:
        print 'Correct!'
    else:
        print 'Not quite -- try again'
\end{sageverbatim}

%In the language of graphs, the number of friends a person has is also
%called the \textbf{degree} of the corresponding \textbf{vertex}.  So
%the command \textbf{G3.degree(5)} accesses the degree (number of
%friends) of vertex (person) 5.  

%
%\begin{tikzpicture}[sibling distance=12em]
%  \node {\pgfbox[center,bottom]{\pgfuseimage{bridges}}}
%    };
%\end{tikzpicture}









