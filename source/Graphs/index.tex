\chapter{Graphs}

\section{Introduction}

As we have found a number of times before, Graph Theory is one of
those areas of mathematics that makes frequent use of words that are
borrowed from everyday English (and from other areas of math), but
with subtle and important differences in meaning that can trip you up
if you're not careful.  Why do mathematicians do this to us?  Because
it helps them understand what they are doing to use metaphors that
guide their intuition.


The kind of graph you're probably already familiar with in math is the
Cartesian graph, a collection of ordered pairs of numbers, laid out on
a grid on paper to make it convenient to display the values of
different mathematical functions.  Below is a graph of the function
$f(x) = mx + b$, a form familiar to many students from high school
algebra.  You can play with the values of $m$ and $b$.

\begin{sageverbatim}
%hide
x = var('x')
@interact
def _(m = (1..10), b=(1..10)):
    b = 2
    c = plot(m*x + b,(x,0,10))
    print m*x + b
    c.show()
\end{sageverbatim}












