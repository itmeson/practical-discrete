\chapter{Graphs}

\section{Introduction}

Let's start with a silly question:  how many friends are there?
What's silly about it?  Well, what does it even mean?  Let's make it
more concrete:

\begin{sageverbatim}
%hide
%auto
M = Matrix ([ [0, 1, 1, 1, 1, 1],   #a 6x6 adjacency matrix
              [1, 0, 1, 0, 1, 0], 
              [1, 1, 0, 0, 1, 1],
              [1, 0, 0, 0, 0, 0],
              [1, 1, 1, 0, 0, 1],
              [1, 0, 1, 0, 1, 0] ] )
G = Graph(M)
G.show()
\end{sageverbatim}

the object above is called a \textbf{graph}.  The numbered points
represent people.  The lines connecting those points represent
friendships.  ``Person 0'' is popular.  They have five friends --
everybody else.  Person 3 is not quite so popular, they only have one
friend.  How many friends do each of the other people have?

\begin{sageverbatim}
%hide
%auto
 
@interact
def data(d0= '0', d1= '0', d2= '0', d3= '0', d4= '0', d5= '0'):
    degrees = [d0, d1, d2, d3, d4, d5]
    degrees = [int(x) for x in degrees]
    d_act = [G.degree(x) for x in G.vertices()]
    if d_act != degrees:
        print "one or more of these is wrong: ", degrees
    else:
        print "All correct."
\end{sageverbatim}


The kind of graph you're probably already familiar with in math is the
Cartesian graph, a collection of ordered pairs of numbers, laid out on
a grid on paper to make it convenient to display the values of
different mathematical functions.  Below is a graph of the function
$f(x) = mx + b$, a form familiar to many students from high school
algebra.  You can play with the values of $m$ and $b$.

\begin{sageverbatim}
%hide
x = var('x')
@interact
def _(m = (1..10), b=(1..10)):
    b = 2
    c = plot(m*x + b,(x,0,10))
    print m*x + b
    c.show()
\end{sageverbatim}












