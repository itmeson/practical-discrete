\chapter{Proof}

If you ask any nonmathematician what mathematics is about, they will
probably say ``Numbers.''  But if you ask a mathematician what
math is about, they will almost certainly say: ``Proof.''

Proof is, unquestionably, the center of mathematics.  It is the bread
\emph{and} the butter.  Why is it so important?  Because, in the words
of *some mathematician* it is the closest to an eternal truth that we
humans can aspire to, down in the here below.

Proofs are also fun.  It is a lot of fun to develop an idea, flesh it
out, and prove it through an airtight argument.

My goal in this chapter is to introduce several styles of proof while
keeping sight of the fun.

Let's start with the simplest kind of argument you can make:
\begin{eqnarray}
 A & \longrightarrow & B \\
 A & & \\
 B & & \\
\end{eqnarray}

The arrow $\longrightarrow$ means ``implies.''  So the first sentence
reads, ``A implies B'', or ``If A is true, then B is also true.''  The
second sentence reads, ``A is true''.  The third line then reveals the
consequence of the previous two -- B must be true as well.

\begin{sageverbatim}
%hide
%auto
print "n=1:\t     1 == 1(1+1)/2\t", 1 == 1*(1+1)/2  
print "n=2:\t 1 + 2 == 2(2+1)/2\t", 1 + 2 == 2*(2+1)/2
\end{sageverbatim}

\begin{sageverbatim}
%hide
%auto
@interact
def _(n=(1..100)):
   summation = " + ".join([str(x) for x in range(1,n+1)])
   product = n*(n+1)/2
   html('Left: $%s\;=\;%s$'%(summation,sum(range(1,n+1))))
   html('Right: $\\frac{n(n+1)}{2}\;=\;\\frac{%s(%s+1)}{2}\;=\;%s$'%(n,n,product))
 \end{sageverbatim}



