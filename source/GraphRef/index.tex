\chapter{GraphRef}

\section{Definitions}

\begin{enumerate}

\item A graph, $G$, is a set of vertices, $V$, and a set of edges, $E$, such that each edge $e \in E$ is associated with an unordered pair of vertices, $(a,b)$.

\item The set of edges of a graph is denoted $E(G)$.

\item The set of vertices of a graph is denoted $V(G)$.

\item A \emph{multigraph} is a graph $G$ in which more than one edge is permitted between two vertices.

\item Two vertices $a, b \in V(G)$ are said to be adjacent if an edge connects them, ie $(a,b) \in E(G)$.

\end{enumerate}

\section{Sage Tools}
\begin{enumerate}

\item Graph.  The simplest way to construct a moderate sized graph in Sage is
using an adjacency matrix, $A_{ij}$, in which two vertices $(i,j)$ are
connected by an edge if the entry $A_{ij}$ is non-zero.  A multigraph can be
constructed with non-unit matrix entries, while the graph can be restricted to
a case of a simple graph if every entry is either $1$ or $0$:

\begin{sageverbatim}
M = Matrix ([ [0, 1, 1, 1, 1, 1],   #a 6x6 adjacency matrix
               [1, 0, 1, 0, 1, 0],
               [1, 1, 0, 0, 1, 1],
               [1, 0, 0, 0, 0, 0],
               [1, 1, 1, 0, 0, 1],
               [1, 0, 1, 0, 1, 0] ] )

eList = []
for e in G.edges():    
    if e[0] == 0 or e[1] == 0:
        eList.append((e[0],e[1],None))

graph_highlight_edge_set(G, eList, color='red', default='green')
        
print G.edges()
\end{sageverbatim}


\end{enumerate}
